\documentclass{article}
\usepackage[utf8]{inputenc}
\usepackage[T1]{fontenc}
\usepackage[english]{babel}
\usepackage{graphicx}
\usepackage{amsmath}
\usepackage{amssymb}
\usepackage{hyperref}
\usepackage{epsf}
\usepackage{float}
\usepackage{geometry}
\geometry{hmargin=3.5cm, vmargin=2.5cm}
\usepackage[squaren]{SIunits}
\usepackage{listings}
\usepackage{color}
\definecolor{mygreen}{RGB}{70, 180, 90}
\definecolor{mylilas}{RGB}{255, 117, 45}
\definecolor{cadr}{rgb}{0.89, 0.0, 0.13}
\graphicspath{{DWGs/}}
\usepackage{graphicx}
\usepackage{wrapfig}
\usepackage{graphicx}
\usepackage{multicol}
\usepackage{enumitem}

\begin{document}

\lstset{language=Python,
    breaklines=true,
    morekeywords={matlab2tikz},
    keywordstyle=\color{blue},
    morekeywords=[2]{1}, keywordstyle=[2]{\color{black}},
    identifierstyle=\color{black},
    stringstyle=\color{mylilas},
    commentstyle=\color{mygreen},
    showstringspaces=false,
    numbers=left,
    numberstyle={\tiny \color{black}},
    numbersep=9pt,
    emph=[1]{for,end,break},emphstyle=[1]\color{red},
}


\section*{Reptiles classification example - introduction to labeling and measures} \label{chap:intro}

\begin{equation}
\mathbf{D} = \mathbf{A} \mathbf{B} \mathbf{C}
\label{eq:decomposition}
\end{equation}


%\begin{figure}[H]
%\centering\includegraphics[width=7cm]{Correlation}
%\caption{Symmetry in the correlation coefficient.}				
%\label{fig:correlation}
%\end{figure}

\lstinputlisting[label=similarity_mat, caption=Code for reptile data. \emph{reptiles.py}]{reptiles.py}



\setlength{\parindent}{0.5cm}

\begin{thebibliography}{50}

\bibitem{MIT} https://www.youtube.com/watch?v=h0e2HAPTGF4

\end{thebibliography}

\end{document}
