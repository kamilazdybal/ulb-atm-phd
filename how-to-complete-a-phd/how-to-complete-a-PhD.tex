\documentclass[10pt,twocolumn]{article}
\usepackage{geometry}
\geometry{verbose,headsep=3cm,tmargin=2.5cm,bmargin=2.5cm,lmargin=2.0cm,rmargin=2.0cm}
\usepackage{graphicx}
\usepackage{xcolor}
\usepackage[font=small]{caption}
\usepackage{amsmath,amssymb,latexsym}
\usepackage{marvosym}
\usepackage{url}
\usepackage{lipsum}
\usepackage{bm}
\usepackage{float}
\usepackage[english]{babel}
\usepackage{subcaption}
\usepackage{subfloat}
\usepackage{epsf}
\usepackage{float}
\usepackage{mathpazo}
\usepackage{pifont}
\usepackage{wrapfig}
\usepackage{multicol}
\usepackage{enumitem}
\usepackage[hidelinks]{hyperref}
\usepackage{booktabs}
\usepackage{xcolor}
\usepackage{framed}
\usepackage[utf8]{inputenc}
\graphicspath{{plots/}}
\usepackage{framed}
\usepackage{textcomp}
\usepackage{braket}
\newcommand{\highlight}[1]{%
  \colorbox{orange!50}{$\displaystyle#1$}}
% Default fixed font does not support bold face
\DeclareFixedFont{\ttb}{T1}{txtt}{bx}{n}{10} % for bold
\DeclareFixedFont{\ttm}{T1}{txtt}{m}{n}{10}  % for normal

% Custom colors
\usepackage{color}
\definecolor{deepblue}{rgb}{0,0,0.5}
\definecolor{deepred}{rgb}{0.6,0,0}
\definecolor{deepgreen}{rgb}{0,0.5,0}

\usepackage{listings}

% Python style for highlighting
\newcommand\pythonstyle{\lstset{
language=Python,
basicstyle=\ttm,
otherkeywords={self},             % Add keywords here
keywordstyle=\ttb\color{deepblue},
emph={MyClass,__init__},          % Custom highlighting
emphstyle=\ttb\color{deepred},    % Custom highlighting style
stringstyle=\color{deepgreen},
frame=tb,                         % Any extra options here
showstringspaces=false            % 
}}


% Python environment
\lstnewenvironment{python}[1][]
{
\pythonstyle
\lstset{#1}
}
{}

% Python for external files
\newcommand\pythonexternal[2][]{{
\pythonstyle
\lstinputlisting[#1]{#2}}}

% Python for inline
\newcommand\pythoninline[1]{{\pythonstyle\lstinline!#1!}}
% Document font:
\usepackage{charter}


\begin{document}

%%% HEADER --------------------------------------------------------------
% ------------------------------------------------------------------------

\twocolumn[{
\begin{@twocolumnfalse}

  \begin{center}
%\textcolor{lgray}
    \vskip-5em

    \hfill
    \fontsize{10}{10}\selectfont {\textit{Bruxelles, 2023}}
    \vskip2ex
	\vspace{5ex}
    \fontsize{20}{10}\selectfont {How to complete a Ph.D.}
      \vspace{1ex}

  \noindent%
    
\vskip1ex

{\rule{\textwidth}{0.5pt}}

  \end{center}
  
    \fontsize{7}{10}\selectfont {This work is licensed under the Creative Commons Attribution-NonCommercial-ShareAlike 4.0 International (CC BY-NC-SA 4.0) license.}

\vspace{6mm}

\end{@twocolumnfalse}
}]

%%% HEADER END -----------------------------------------------------------
% ------------------------------------------------------------------------

\vspace{10mm}

\setlength{\parindent}{0cm}

\fontsize{14}{10}\selectfont {Kamila Zdybał}

\vspace{2mm}

\fontsize{8}{10}\selectfont {\textit{\href{https://kamilazdybal.github.io/}{kamilazdybal.github.io}, kamila.zdybal@gmail.com}}

\,\,

Please feel free to contact me with any suggestions, corrections or comments.

\section*{Preface}

\setlength{\parindent}{0em}
\setlength{\parskip}{0.5em}

\small

Doing a Ph.D. has been an immensely rewarding journey for me. It has been really difficult at times, and I mean sobbing-at-my-desk difficult. And at other times, I felt like conquering the top of the world! I’m grateful for every experience from my Ph.D. Those experiences taught me so much, made me more resilient and more confident. This note is a collection of items that helped me complete my Ph.D. and thrive in a Ph.D. program. I’m sharing those with you in hopes that they will help you complete your own journey! You can absolutely end your Ph.D. with a smile on your face and a great sense of accomplishment!

\section*{Keywords}

\textit{graduate school, Ph.D., academia, academic advisor}

\subsection*{1. Ask questions. Early and often.}

Let’s face the tough truth: when you start your Ph.D., you know close to nothing about doing research or about your research discipline. That’s normal and expected. But you’re going to have to grow during your Ph.D. And in order to grow, you’re going to have to convey to your advisor what it is that you don’t know or don’t understand. So, ask questions early and often.

I’ve learned this the hard way. During the first year of my Ph.D., I was so scared that my advisors would find out how much I didn’t know that I deliberately avoided asking questions and sometimes even avoiding speaking to them altogether. At some point, I noticed that I wasn't making as much progress as I would have liked. I realized that the only way I can learn to do good research is if I explicitly tell my advisors: \textit{Hey, I don’t understand that. Could you teach me?} And you know what? My advisors were entirely okay with teaching me things! Did I sometimes ask them undergraduate-level stuff that I should have known by now? Yes. Did they think in their minds: \textit{Wow, she really should have known that by now!} I don’t know; probably yes! Did whatever they thought in their minds have any negative effect on the outside world? Not at all! They were still happy to meet with me and help me out. That was the only way for our research to move forward. So, don’t be scared of your advisor; their job is to boost your abilities. The more open you are with them about what you’re struggling with, the more efficient their help will be. And sometimes you’ll find that that question you feel embarrassed to ask is actually one of the unknowns of your research that even your advisor doesn’t yet know the answer to!

\subsection*{2. Set up weekly meetings with your advisor.}

I consider weekly meetings with one of my advisors the biggest contributor to the successful completion of my Ph.D. So many times, just the act of speaking my current research struggles out loud to my advisor helped me gain clarity on the path to take next or helped me shape my way of thinking about a problem. Sometimes, thanks to our weekly discussions, it became clear to me how to write a section in our paper explaining our results. A weekly meeting keeps up the momentum of your work and holds you accountable. There were times when I felt slumps in my motivation and felt like slacking off, but I realized that I couldn't come to the meeting empty-handed. Having that meeting looming over me kept pushing me towards completing tasks and doing the necessary analysis, despite the lack of motivation.

Two research-related things that you should aim to take away from every meeting are:

\begin{enumerate}
\item \textbf{Resolution of any uncertainties, difficulties, or queries.} Your advisor should be able to provide direction whenever problems arise. Signal those problems to your advisor and let them help you.
\item \textbf{Clarity on what needs to be done next (in the days until your next meeting)}. If, after the meeting, you feel energized to tackle the next task, it means the meeting was effective. If, after the meeting, you feel confused or clueless, it means that something needs to improve in the communication between you and your advisor. Signal this issue to your advisor and try to come up with a communication strategy that allows you to gain the necessary clarity.
\end{enumerate}

But there’s more than that. A chance to have a weekly meeting with a brilliant academic is also a wonderful privilege! Imagine getting an hour a week of one-on-one, free chat with a real scientist. How cool is that? You can ask them all sorts of interesting questions and brainstorm research ideas. You will profit enormously from observing the way of thinking of a more senior academic. So, schedule a regular meeting time with your advisor at the very beginning of your Ph.D. and stick to it.

\subsection*{3. Understand that your advisor is busy and act accordingly.}

Your advisor is likely juggling a few research projects at a time, teaching responsibilities, travel, giving talks, and any management or service that the university requires from them. And on top of that, they’re trying to have a life and spend time with their family (your advisor is a human too!). Your advisor still cares for you and has your best interests in mind. They just might not have the capacity to help you immediately and will often need you to help them to help you. Let me give you some examples.

\begin{itemize}

\item Your advisor will be forgetful. Therefore:

\begin{itemize}
\item Stay organized with your results and ideas so that they can easily be retrieved and recalled.
\item Come prepared to your weekly meetings. Plan to spend the first 5 minutes of the meeting to remind your advisor where you’ve left off the previous week. Have a list of items to discuss during the meeting. \item During the meeting, take notes on what needs to be done.
\item Stay organized with important deadlines. Deadlines for conference submissions, returning rebuttals, or university re-enrollments shouldn’t be your advisor’s responsibility.
\end{itemize}

\item Your advisor will struggle to reply to your messages/emails. Therefore:

\begin{itemize}
\item Shorten your email messages to the necessary minimum. Whenever possible, aim to send short messages asking for a very specific feedback. Instead of sending a 10-page paper asking for comments, take a screenshot of a specific result and ask: \textit{Do you think that with this result I can make a claim in the paper that $\dots$?}
\item Don’t contact your advisor during weekends or holidays. They have the right to take time off and spend time with their family too!
\end{itemize}

\item Your advisor will struggle to maintain an overview of your work. Therefore:

\begin{itemize}
\item Try to convey a story with the results that you show your advisor on a weekly basis. A storyline lets you convince your advisor of certain outcomes in support of your hypotheses. Stories are also naturally easier to follow and memorize. They help your advisor to “connect the dots”.
\item Prepare efficient slides/visuals for your weekly meetings where you explain or even re-explain important theoretical concepts or ideas.
\item Remind your advisor of the big picture: What hypothesis are you currently trying to prove? What results or datasets do you aim to collect in the next months? How would you present the main storyline in the paper that you’re writing?
\end{itemize}

\end{itemize}

Of course, a significant level of neglect from your advisor is unacceptable and should be resolved as soon as possible. This includes if you haven’t received feedback on your written work for a month or longer or if your advisor hasn’t met with you for a month or longer. However, the important thing is that you understand that when it’s been a week and you haven’t heard back from your advisor or when your advisor seems to have lost connection with your research when they meet with you, it’s not about you. It’s simply about how demanding an academic job is nowadays.

\subsection*{4. Take initiative. Be proactive.}

You need to take responsibility of your research project. There are two main reasons for this:

The purpose of a Ph.D. training is for you to develop skills that will allow you to lead a research project yourself in the future. (Some people call this becoming an “independent researcher”.)
Your advisor likely doesn’t have the time to take full responsibility of your research project! (See the previous advice.)
So, for example, instead of waiting for your advisor to tell you – \textit{How about we write a paper on XYZ?} – monitor your results and find possible venues where you might want to send them to at some point. Discuss those paper ideas with your advisor. Suggest a sketch for the first draft. Instead of waiting for the next weekly meeting to hear your advisor suggest the next checks to be performed, try to think if there’s any extra checks that you could do until then that would be helpful for your upcoming discussion. As you progress in your Ph.D., you’ll find that more often than not you’re able to make good decisions on what needs to be checked next! Remember that it’s your Ph.D. and you are at the driver seat of your research. Your advisor is not your manager, but rather a resource to help you complete your Ph.D. You largely get to decide where your research takes you!

\subsection*{5. Learn to accept and welcome failure.}

I have a personal hypothesis that accepting failure is harder for students who start a Ph.D. right after their undergraduate degree, and less difficult for students who had some research or work experience before starting a Ph.D. (I’ll be curious to hear what research says about that hypothesis though!) If failure is hard for you to handle, that’s normal! There are reasons for why it’s hard. When we’re at school and we get a bad grade, fail a test, or fail a class, it can be very definite. Often we’re not even allowed for a second chance to improve and try again; the bad grade just stays there until the end of the school year. Sometimes our teachers don’t take the time to explain to us what we’ve done wrong, or where are the flaws in our understanding. Other times, we might be given a chance to re-take a test, or re-take a class, but the consequences can still be enormous. At my university, failing a course meant that the student was forbidden to take certain other courses next semester, and the debt of missed courses kept accumulating over the undergrad years. This caused some of my classmates to be severely overtime, and some of them to drop out. The good news is: the real life is quite different! The Ph.D. journey is different too. Failure will come your way sooner or later in your Ph.D. and that’s alright and expected! Unlike at school, you will be given the chance to learn, improve, and try again. Other researchers will give you feedback and advice on how to improve your work and try again. Failures won’t deprive you of opportunities, or at least not for long. Failures will not be listed on your Ph.D. diploma!

Here are some of the failures that you can expect during your Ph.D. and some ways of handling them:

\begin{itemize}
\item \textbf{Paper rejections will come your way.} They happen to all of us. Heck, they even happen to senior and renowned scientists! I still remember my first paper rejection. It was a Saturday and I spent the whole morning crying in bed. The second paper rejection was easier. When the third paper rejection came my way, I went outside to breathe some fresh air and 15 minutes later I texted my advisors: \textit{What do you think about sending our paper to journal XYZ instead?} The upshot is: You will learn to cope with it, or as one professor once told me: \textit{You will grow a thick skin.} Does it still suck to get your paper rejected? Sure. But it’s not the end of your research! Here’s what you can do after the rejection:

\begin{itemize}
\item Take the opportunity to learn and improve your work. Rejections after peer review give you access to reviewers’ comments. Read those comments for any suggestions and ideas on improving your work, and implement them before re-submitting your paper. Looking back at my rejected papers, I’m very glad that they were rejected! I can now see the ways in which my work wasn’t up to par and my statements weren’t thoroughly supported by the results. If those papers were accepted at the state that I first submitted them, I would feel embarrassed today! Reviewers saved me from that embarrassment, and, more importantly, saved the research community from research that was incomplete and might have been misleading. That’s the role of a well-functioning peer review process. To paraphrase Oscar Wilde: \textit{There is only one thing in the world worse than having your paper rejected and that is having your paper retracted!}

\item There is a great piece of advice that I once found online: \textit{If your papers always get accepted, it means you’re aiming too low.} This means that, as a researcher, you should aim to have your papers rejected every once in a while! Otherwise you’ll never know which impactful journals your papers might be accepted in.
\end{itemize}

\item \textbf{You will get negative results}, \textit{i.e.}, you will try to prove something and the results you’ll get just won’t show any evidence in support of your hypotheses.

\begin{itemize}
\item Learn to fail quickly. Aim to perform quick and cheap experiments during your Ph.D. If something doesn’t seem to work, move on. A Ph.D. is time-bounded, and you might not have the luxury to stubbornly pursue an idea that doesn’t seem to be promising for months. See if you can implement a modification to the idea. Be persistent, but also be mindful of your time frame. A good advisor should be able to tell you: \textit{Try this for another 2 weeks and if that doesn’t work we’ll revise our plan and try something else.} When you give up on an idea, keep it in your notes for your future research. One day, a new insight might help you improve your old idea.
\item Negative results still teach us things. Discuss those results with your advisor. It is important that your advisor is aware of the negative result as that will save them time (for example, they won’t schedule another student to work on this, or they will know not to write a grant proposal based on that idea). Your advisor should also be able to see if there is an opportunity to write-up your negative result anyway. My personal opinion is that we as researchers still don’t share enough of “research failures”. I suspect that plenty of grant money could be saved if research groups communicated failed ideas with each other! Which brings me to…
\item Disseminate your negative result. You can do this in all sorts of ways. Talk about the “failed idea” at conferences, with colleagues and collaborators. Write a paper about it – there are research journals specifically dedicated to publishing negative results. Otherwise, when you write your next research paper, mention that you tested XYZ and it didn’t work. Your negative results, although not as satisfying as positive results, can still be equally helpful to the research community.
\end{itemize}


\item \textbf{You will find out that someone else had the same idea as you did and has already published it.} This happens all the time in research. Things to do when that happens to you:

\begin{itemize}
\item Discuss this situation with your advisor. Your advisor should be able to provide you guidance on what to do next, how to present your research in a way that still differentiates you from the existing work.
Chances are, you can still bring some new perspective to the story. Even if it’s not as impactful of a contribution as: \textit{We discovered XYZ}, it can still be: \textit{We applied XYZ to this new problem}, or: \textit{We improve the method XYZ by introducing this modification}. In fact, in moooost of the published research, researchers develop a tiny, incremental advance to the existing state of the art. It’s very rare that researchers present a breakthrough that no one has ever thought about before! Keep in mind that you don’t need a breakthrough to earn a Ph.D.!
\end{itemize}

\item \textbf{You will get grant and award rejections.}

\begin{itemize}
\item Take the opportunity to learn. Every grant proposal, research statement, nomination letter that you or your advisor writes teaches you how to disseminate, showcase, and think about your research. Even if your research proposal is rejected, the time that you took to prepare it was a time well-spent. Just by the act of writing your proposal you can gain clarity on what you’d like to develop in the coming months or years.
\item Know that there’s a lot of luck that goes into getting a grant or an award, and luck is completely outside of your control! A rejection might not at all be a reflection of how good you and your research are.
\end{itemize}
\end{itemize}

I wish you to have your first failures as early as possible in your Ph.D.! You will learn a lot from them.

\subsection*{6. Write code to automate your work.}

Your future-self will thank you for any automation that you do today. I usually find that the time spend to write a Python script to automate a task saves me ten times more time later.

Say you need to run a post-processing tool after you collect files with some results. Say that your results files arrive incrementally as you decide to test new scenarios. You can manually check if a given scenario has finished running and then run your post-processing tool. Or, you can parameterize your cases and automatically check if a results file for a given case is already there (the Python way: \texttt{os.path.exists()}) and if yes, run the post-processing, if not, just keep going to the next case. You can run such script once a day, or, even better, you can schedule it as a \texttt{cron} job.

The same thing goes with other dull tasks such as renaming many files, organizing files, running various scenarios, generating figures. It’s worth spending time at the beginning of your Ph.D. to learn how to do those things (\textit{e.g.}, in Python). Learn basic operation on files (creating, renaming, appending, saving, moving), learn operation on strings (regular expressions), and learn list comprehensions. For example, say you need to generate a header for a \LaTeX\  table that says Case-\# where \# is an even number from 2 to 20. With this list comprehension in Python:

\texttt{' \& '.join(['Case-' + str(i) for i in range(1,21) if i\%2==0])}

you’ll get:

\texttt{Case-2 \& Case-4 \& Case-6 \& Case-8 \& Case-10 \& Case-12 \& Case-14 \& Case-16 \& Case-18 \& Case-20}

that you can copy and paste into your \LaTeX\  file. Now imagine that you need a similar header but for one million numbers!

You can find more automation ideas in my YouTube tutorials, \href{https://www.youtube.com/playlist?list=PL7gWbAt3_3KEuRQfwFeI_RH3EZr87nslf}{\textcolor{deepblue}{\textbf{\textit{Python for academics}}}}.

\subsection*{7. Write regularly.}

As academics, part of our job is to disseminate our research. There would be no point in making a new discovery if no one ever finds out about it, right? Writing for an audience is therefore carved in our job description. Throughout your Ph.D., you’re going to have to learn academic writing. You’re not expected to be good at it at the start, but, like with every other skill, you’re going to be making progress in it. The only way to learn writing is to write. As you write more, you’ll start getting better at it. Schedule time to write regularly. It can be one paragraph for your next paper describing your current results. It can be one paragraph describing theory for your dissertation. It can be an idea for a conference abstract. Write throughout your Ph.D. and learn the rules of good academic writing. Check out \href{https://kamilazdybal.notion.site/Improving-your-academic-writing-42da2498f8f14e3a9c9e20a068cbe472}{\textcolor{deepblue}{\textbf{\textit{this Notion page}}}} where I collected many resources on improving your academic writing.

\subsection*{8. Learn to make effective visuals.}

Efficient communication of your research and your ideas not only increases the impact of your work and makes you more memorable, but it also saves you and your audience a lot of time. In particular, effective visualization can help you convey your ideas with clarity and speed. This applies to presentations that you give throughout your Ph.D. but also to your weekly discussions with your advisor or even email exchanges with collaborators. Spend time at the beginning of your Ph.D. to learn the rules of creating clear figures. Give some thought when preparing visual aids and slides. Implement efficient visualization in your journal articles and your dissertation. You will not only help yourself gain clarity, but, perhaps more importantly, you will help your audience grasp the ideas faster. Imagine that 100 researchers will read your paper. Imagine that with poor explanation or visualization, they will each have to spend extra 15 minutes trying to understand your methodology. By implementing the rules of clearer communication, you’re saving 25 hours of other people’s grant money. That’s around 3 working days. Now imagine that your great paper will be read by 1000 researchers (which I hope will be the case!) Now you’re saving over a month worth of taxpayers’ money! Worth spending that extra hour or two polishing that figure, huh?

A good diagnosis of the clarity of your communication is answering the question: \textit{Does my advisor frequently need to ask for clarifications while I present?} If the answer is yes, it’s likely that you’re presenting your results in a confusing way. Here’s some items that you can think of implementing:

\begin{itemize}
\item Keep your notation consistent. Make it clear what variables you’re plotting or what variable is being visualized in a colorbar. Don’t change the notation halfway through your slides.
\item Make visual pointers to the most significant results supporting/rejecting your hypothesis.
\item In your informal communication with your advisor, it’s okay to use unscientific terms like “bad result” or “great result”. When you work closely with your advisor, you both know what a “bad result” or a “good result” means for you. Such human-readable statements, especially supported by appropriate coloring, help increase the speed of conveying a message. Just remember to formalize your wording in a journal article, or at a conference talk.
\item Make reasonably big figures. Increase font sizes, increase marker sizes, increase line thickness. Slide space costs nothing, so use it well!
\item Remove visual clutter.
\item For any materials that might be printed, use colors and colormaps that will be clear in black and white.
\end{itemize}

You can also check out my earlier note on \href{https://kamilazdybal.github.io/jekyll/update/2022/08/07/making-presentations.html}{\textcolor{deepblue}{\textbf{\textit{Making effective presentations}}}} for more ideas for your consideration.

\subsection*{9. Understand that once you get into the unknowns of research, even your advisor doesn’t have all the answers.}

Doing research is doing what no one has done yet. The purpose of your Ph.D. is to produce new knowledge. Your advisor can often provide advice or helpful perspective, but there will be times when even your advisor won’t have a good intuition, won’t know a particular theory or a method that could be helpful. Sometimes, you’re going to have to spend time to check, debug, and test things, so that both you and your advisor learn the answer to a particular question. Other times, you’re going to have to engage other academics to help you out. Those could be other researchers in your department, other scientists worldwide, or people you meet at conferences. A phrase I’ve once heard that I like is: \textit{Don’t put all your eggs in the supervisory basket.} I think it summarizes my point quite well. Use resources other than your advisor to move your research forward.

\subsection*{10. Schedule time for thought work.}

Sometimes our workday as Ph.D. students looks like this: we come to the lab, open our laptops, start running simulations, collect results, post-process results, read, write, close our laptops and go home. But we tend to forget that simply thinking is part of our job too. Sometimes things simply need thinking through, away from our computers or equipment, and we need to schedule time to do that kind of work. I find that I need a quiet and dedicated space to do the thought work. It’s hard to do the thought work when in the lab, when there are other people around talking. You might therefore need to schedule that work when at home when you can fully focus and nothing in your surrounding can distract you. Having a whiteboard, or pen and paper, can help. Types of research problems that frequently benefit from thought work:

\begin{itemize}
\item When you need to nail down how a certain algorithm/method you’re developing needs to work.
\item When you need to understand an existing method.
\item When you need to plan steps for an experiment to execute.
\item When you need to come up with cases or scenarios that you need to test/run.
\item When you need to recall an important theory.
\end{itemize}

Being a researcher is that extraordinary occupation, where a person from outside can catch you simply sitting in your chair, and you are, in fact, working!

\subsection*{11. Take breaks.}

Your mind and your body need some time off from your research, time off from your lab bench, time off from your desk. This is primarily to keep you healthy and happy. But also, no one can be focused for 10 hours a day! Research shows, and my personal experiments confirm that, that we can have at most ~4 hours of focused work a day. Any work that’s done beyond that can still be useful, but it’s not a deeply focused work. Scheduling short breaks throughout the day lets you have those 4 hours of focused work. Scheduling longer breaks from your work, \textit{e.g.}, by taking the weekend off or taking holidays, keeps you energized and keeps your Ph.D. fresh. Most of my workdays, I break the 8-hour workday into 1- or 2-hour long chunks, and go out for a walk, go out for a run, cook lunch listening to a podcast, or do laundry in between those chunks. I often use the pomodoro timer too and aim to be at my desk for 30 minutes, and stand up and move for 5 minutes. Taking those short breaks helps me be much less exhausted at the end of the day, compared to days when I come to the lab in the morning and sit at my desk for 8 hours. There’s also a side-effect of taking breaks. Some of my most precious research ideas or solutions to coding problems came to me while I was away from my desk and enjoying a run in the forest or a long meditation.

\subsection*{12. Accept that some days or even some weeks will be unproductive. And that’s alright.}

In every four or five year long Ph.D. candidature there are days when you get nothing done, for whatever reason. Don’t beat yourself for that, and don’t feel guilty for that. It’s normal and happens to everyone, even the “best” students. No one can be at 100\% energy 5 days a week for four or five years. Every once in a while, you’re going to come to your weekly meeting with your advisor and say: \textit{I wasn’t able to do much work these days, I think I don’t have much to discuss today.} There were certainly times when I had to say that to my advisors. It sucked to have to say that. But that’s just how life is. And, as long as this doesn’t happen every week, they will understand.

\subsection*{13. Prioritize your growth and your own goals.}

Once you get your hands dirty with research, there can be months when all you do is try to nail down a particular problem, write extra code, wait for cases to finish running, produce plots, and start all over again. It’s easy to get into a rut of research and forget why you wanted to pursue a Ph.D. in the first place. Maybe you were hoping to learn a particular skill or theory? Read a particular textbook? Take a particular class? Or, write that “\textit{How to complete a Ph.D.}” guide for future generations that you’ve been putting off for so long! Throughout your Ph.D., try to remember about your own growth, your own goals, what's important to \textbf{you}, and what you wanted to get out of it for yourself. Ideally you should aim to have that “me-time” every workday – you could, for example, plan that the first hour of every day is for learning and experimenting that you truly want for yourself. In other words, you shouldn’t just aim to make your advisor happy! A little of such selfishness along the way is going to do wonders to your sense of purpose and growth! It might even make the rest of your workday much more pleasant and energized.

\section*{My last two cents}

Remember that no Ph.D. student is a perfect student but also no advisor is a perfect advisor! Accept that you’ll make mistakes or fall behind, and accept that other times your advisor will make mistakes or fall behind. There’s many things that you have control over in your Ph.D. journey. You can improve your experience enormously with the ideas that I gathered here. If you’re still hungry for more guidance, watch these two YouTube videos:

\begin{itemize}
\item \href{https://www.youtube.com/watch?v=WNzwwTYysTU}{\textcolor{deepblue}{\textbf{Who are the most challenging students to supervise}}}
\item \href{https://www.youtube.com/watch?v=WokV1JYgqZo}{\textcolor{deepblue}{\textbf{The characteristics of the best PhD students}}}
\end{itemize}

I kept re-watching these videos during my Ph.D. to remind myself of what student not to be towards my advisors, and what student to try to become.

I hope that your Ph.D. adventure will be a fulfilling and inspiring experience to you. Good luck!

\end{document}